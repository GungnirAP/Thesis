\documentclass[letterpaper,11pt]{article}
\usepackage{tabularx} % extra features for tabular environment
\usepackage{amsmath}  % improve math presentation
\usepackage{amssymb}
\usepackage{graphicx} % takes care of graphic including machinery
\usepackage{multirow}
\usepackage{bbding}
\usepackage{caption}
\usepackage{subcaption}
\usepackage{float}
\usepackage{multicol}
\usepackage{setspace}
\usepackage{booktabs}
\usepackage{realhats}
\usepackage{array}
\usepackage{bbold}
\renewcommand\arraystretch{1.5}
\def\sym#1{\ifmmode^{#1}\else\(^{#1}\)\fi}
\doublespacing
\usepackage[margin=1in,letterpaper]{geometry} % decreases margins
\usepackage{cite} % takes care of citations
\usepackage[final]{hyperref} % adds hyper links inside the generated pdf file
\usepackage{color,hyperref}
\definecolor{darkblue}{rgb}{0.0,0.0,0.4}
\definecolor{darkgreen}{rgb}{0.0,0.3,0.0}
\definecolor{darkred}{rgb}{0.7,0.0,0.0}
\hypersetup{
	colorlinks=true,       % false: boxed links; true: colored links
	linkcolor=darkgreen,        % color of internal links
	citecolor=darkred,        % color of links to bibliography
	filecolor=magenta,     % color of file links
	urlcolor=darkblue         
}
\usepackage{blindtext}
\usepackage{bbm}
    \usepackage{enumitem} 
            \setlist{nosep}
                \usepackage{geometry} 
    \geometry{top=25mm}
    \geometry{bottom=25mm}
    \geometry{left=25mm}
    \geometry{right=25mm}
    
    %Drawing
    \usepackage{physics}
    \usepackage{amsmath}
    \usepackage{tikz}
    \usepackage{mathdots}
    \usepackage{yhmath}
    \usepackage{cancel}
    \usepackage{color}
    \usepackage{siunitx}
    \usepackage{array}
    \usepackage{multirow}
    \usepackage{amssymb}
    \usepackage{gensymb}
    \usepackage{tabularx}
    \usepackage{extarrows}
    \usepackage{booktabs}
    \usetikzlibrary{fadings}
    \usetikzlibrary{patterns}
    \usetikzlibrary{shadows.blur}
    \usetikzlibrary{shapes}
   
   \usepackage{amsthm}
   \usepackage[english]{babel}
   \newtheorem{definition}{Definition}
   \newtheorem{theorem}{Theorem}
   \newtheorem{proposition}{Proposition}
   
   
   \usepackage[natbibapa]{apacite}
   
%++++++++++++++++++++++++++++++++++++++++


\begin{document}

\title{\textsc{Public good and attitude towards ethnic diversity} \\ \vspace{.5cm} \large \textsc{Thesis}}
\author{Ivan Dedyukhin}
\date{\today}
\maketitle

\section{Research proposal}

Ethnic diversity is an essential determinant of economic performance. For example, \cite{Africa} shows that ethnic diversity explains the difference in public policies between countries in Sub-Saharan Africa. These varieties explain lower economic growth in countries with higher ethnic fragmentation. Talking about public goods, some papers argues that the higher ethnic diversity is associated with lower provision of some public goods ( \cite{AlesinaDivision}, \cite{Kenya}, \cite{Indonesia}, \cite{difference}). There is no consensus among scientists about the causes and mechanisms of such a relationship, but some explanations exist. \cite{WhyUndermine} contributed to the mechanism of collective action. In similar conditions, agents from the same ethnic group play cooperative equilibrium, but agents from different groups do not. The reason for this may be the fact that we do not like the other group, and we do not want to do anything with it. My research question is based on the literature from above and sounds like this -- "Does ethnic diversity affect the provision of public goods, or is it only the attitude towards nationalities that matters?".

The answer to this question may have a beneficial influence on public policy towards ethnicities. This question is significant in Russia, mainly because it is multinational. In 2010 the index of ethnolinguistic diversity varied from 0.095 to 0.837 for Russian regions (\cite{Russia}). If the attitude towards heterogeneity matters, the state can start a policy of reconciliation, stop nationalism. If diversity matters, the government may not affect the regional division or the settlement of nationalities. Every answer to the stated research question is interesting for economists and politicians because of its effect on population welfare. 

The first part of my work will be a model. The economy will consist of different ethnic groups with agents who have some initial income to spend on two types of goods -- private and public. Every agent's preferences from the group will be represented by the utility function depending positively on private consumption and public goods, and ethnic heterogeneity. The intuition under the impact of private and public consumption is evident. Ethnic diversity hurts the marginal utility of public goods because its consumption is associated with communication with other groups that we may not like. The heterogeneity is dependent on the agent's attitude towards other groups that consume public goods and the number of such people. Firstly, I want to analyze the private provision of public good, and I expect that more negative attitudes and higher fragmentation will hurt the public good's total production.
What is more interesting, I expect them to be substitutes because, intuitively, I believe that communication with the one person from another group that I hate is the same as communication with the ten persons that I do not like. Secondly, I will analyze the state provision of public goods based on the tax system. The government will maximize social welfare in the simplest case, and I plan to implement the utility function that depends on collected money.

The second part of my paper will be an empirical analysis. I will use the dataset from \cite{Refuges} which includes different measures on the ethnic heterogeneity for the bunch of European countries. In addition, it contains country-level socio-economics data from the World Bank's World Development Indicators. The most crucial part is individual-level data from the European Social Survey because it answers the question about attitudes towards refugees, migrants, and minor and major ethnic groups. These data will be an excellent proxy for attitudes of different ethnic groups to other nations. The identification strategy and the construction of attitudes variables are under development.  

\bibliographystyle{apacite}
\bibliography{reference}

\end{document}