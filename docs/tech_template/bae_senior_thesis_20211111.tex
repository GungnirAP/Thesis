%%%%%%%%%%%%%%%%%%%%%%%%%%%%%%%%%%%%%%%%%
% University Assignment Title Page
% LaTeX Template
% Version 1.0 (27/12/12)
%
% This template has been downloaded from:
% http://www.LaTeXTemplates.com
%
% Original author:
% WikiBooks (http://en.wikibooks.org/wiki/LaTeX/Title_Creation)
%
% License:
% CC BY-NC-SA 3.0 (http://creativecommons.org/licenses/by-nc-sa/3.0/)
%
%%%%%%%%%%%%%%%%%%%%%%%%%%%%%%%%%%%%%%%%%%
%\title{Title page with logo}
%----------------------------------------------------------------------------------------
%	PACKAGES AND OTHER DOCUMENT CONFIGURATIONS
%----------------------------------------------------------------------------------------

\documentclass[12pt]{article}
\usepackage[english, russian]{babel}
\usepackage[utf8x]{inputenc}
\usepackage{amsmath}
\usepackage{graphicx}
\usepackage[colorinlistoftodos]{todonotes}
\usepackage{setspace}
\usepackage[authoryear]{natbib} %this package helps set the right citation style
\usepackage{fullpage} %sets appropriate margins
\usepackage{geometry}
\geometry{
 a4paper,
 total={210mm,297mm},
 left=30mm,
 right=20mm,
 top=20mm,
 bottom=30mm,
 }
\begin{document}
\bibpunct{(}{)}{;}{a}{,}{,}
 \onehalfspacing
\begin{titlepage}

\newcommand{\HRule}{\rule{\linewidth}{0.5mm}} % Defines a new command for the horizontal lines, change thickness here

\center % Center everything on the page
%\vspace*{-3.5cm}
%----------------------------------------------------------------------------------------
%	HEADING SECTIONS
%----------------------------------------------------------------------------------------

\textsc{правительство российской федерации}\\
\textsc{федеральное государственное автономное образовательное учреждение высшего образования}\\
\textsc{"Национальный исследовательский университет}\\
\textsc{"Высшая школа экономики"}\\[1cm] % Name of your university/college
%\textsc{\Large Major Heading}\\[0.5cm] % Major heading such as course name
%\textsc{\large Minor Heading}\\[0.5cm] % Minor heading such as course title

\textsc{негосударственное образовательное учреждение}\\
\textsc{высшего образования}\\
\textsc{"российская экономическая школа" (институт)}\\[2cm]

\textsc{\textmd{выпускная квалификационная работа}}\\[0.5cm]
%----------------------------------------------------------------------------------------
%	TITLE SECTION
%----------------------------------------------------------------------------------------
\begin{spacing}{1.5}
\HRule \\[0.4cm]
{ \huge \bfseries Название вашей выпускной квалификационной работы}\\[0.4cm] % Title of your document
\HRule \\[0.5cm]
\end{spacing}
%----------------------------------------------------------------------------------------
%	AUTHOR SECTION
%----------------------------------------------------------------------------------------

%\textsc{\textit{\large Бакалаврская программа}}\\[0.2cm]
\textsc{\textit{\large Программа Бакалавр экономики}}\\[0.2cm]
\textsc{\textit{\large Совместная программа по экономике НИУ ВШЭ и РЭШ}}\\[2cm]


\begin{minipage}{0.4\textwidth}
\begin{flushleft} \large
\emph{Автор:}\\
 \textsc{И.А. Иванов} % Your name
 \vspace{2.5cm}
\end{flushleft}
\end{minipage}
~
\begin{minipage}{0.4\textwidth}
\begin{flushleft} \large
\emph{Научный руководитель:} \\
\textsc{В.П. Сидоров} \\% Supervisor's Name
\vspace{1cm}
\emph{Консультант:} \\
\textsc{И.С. Петров} % Consultant's Name (if applicable)

\end{flushleft}
\end{minipage}\\[3cm]



%----------------------------------------------------------------------------------------
%	DATE SECTION
%----------------------------------------------------------------------------------------

{\large Москва, 2021 г.}%\\[2cm] % Date, change the \today to a set date if you want to be precise

%----------------------------------------------------------------------------------------
%	LOGO SECTION
%----------------------------------------------------------------------------------------

%\includegraphics{logo.png}\\[1cm] % Include a department/university logo - this will require the graphicx package

%----------------------------------------------------------------------------------------

\vfill % Fill the rest of the page with whitespace

\end{titlepage}


\begin{abstract}
Здесь размещается аннотация работы. Здесь размещается аннотация работы. Здесь размещается аннотация работы. Здесь размещается аннотация работы. Здесь размещается аннотация работы. Здесь размещается аннотация работы. Здесь размещается аннотация работы. Здесь размещается аннотация работы. Здесь размещается аннотация работы. Здесь размещается аннотация работы. Здесь размещается аннотация работы. Здесь размещается аннотация работы. Здесь размещается аннотация работы. Здесь размещается аннотация работы.
\end{abstract}

\selectlanguage{english}
\begin{abstract}
Your abstract in English. Your abstract in English. Your abstract in English. Your abstract in English.Your abstract in English. Your abstract in English. Your abstract in English.Your abstract in English. Your abstract in English. Your abstract in English. Your abstract in English.Your abstract in English. Your abstract in English. Your abstract in English.Your abstract in English. Your abstract in English. Your abstract in English.Your abstract in English. Your abstract in English. Your abstract in English.
\end{abstract}


\newpage
\selectlanguage{russian}
\tableofcontents

\selectlanguage{russian}
\newpage


\section{Введение}

\par Здесь будет введение. Здесь будет введение. Здесь будет введение. Здесь будет введение.Здесь будет введение. Здесь будет введение. Здесь будет введение. Здесь будет введение.Здесь будет введение. Здесь будет введение. Здесь будет введение.Здесь будет введение. Здесь будет введение. Здесь будет введение. Здесь будет введение.Здесь будет введение. Здесь будет введение. Здесь будет введение.Здесь будет введение. Здесь будет введение. Здесь будет введение. Здесь будет введение.Здесь будет введение. Здесь будет введение. Здесь будет введение.Здесь будет введение. Здесь будет введение. Здесь будет введение. Здесь будет введение.Здесь будет введение. Здесь будет введение. Здесь будет введение.Здесь будет введение. Здесь будет введение. Здесь будет введение.

\section{Модель}


\subsection{Основные предпосылки}

Основные предпосылки. Основные предпосылки.
Основные предпосылки.
Основные предпосылки.
Основные предпосылки.
Основные предпосылки.
Основные предпосылки.
Основные предпосылки.
Основные предпосылки.
Основные предпосылки.
Основные предпосылки.
Основные предпосылки.
Основные предпосылки.
Основные предпосылки.


\subsection{Анализ равновесий}

Анализ равновесий. Анализ равновесий. Анализ равновесий. Анализ равновесий. Анализ равновесий. Анализ равновесий. Анализ равновесий. Анализ равновесий. Анализ равновесий. Анализ равновесий. Анализ равновесий. Анализ равновесий. Анализ равновесий. Анализ равновесий. Анализ равновесий. Анализ равновесий. Анализ равновесий. Анализ равновесий. Анализ равновесий. Анализ равновесий. Анализ равновесий. Анализ равновесий. Анализ равновесий. Анализ равновесий.



\section{Заключение}

Здесь будет заключение. Здесь будет заключение. Здесь будет заключение. Здесь будет заключение.Здесь будет заключение. Здесь будет заключение. Здесь будет заключение.Здесь будет заключение. Здесь будет заключение. Здесь будет заключение.Здесь будет заключение. Здесь будет заключение. Здесь будет заключение.Здесь будет заключение. Здесь будет заключение. Здесь будет заключение.Здесь будет заключение. Здесь будет заключение. Здесь будет заключение.Здесь будет заключение. Здесь будет заключение. Здесь будет заключение.Здесь будет заключение. Здесь будет заключение. Здесь будет заключение.Здесь будет заключение. Здесь будет заключение. Здесь будет заключение.

% Commands to include a figure:
%\begin{figure}
%\centering
%\includegraphics[width=0.5\textwidth]{frog.jpg}
%\caption{\label{fig:frog}This is a figure caption.}
%\end{figure}


\newpage
\section{Приложение}

Рассмотрим выражение:
$$S_n = \frac{X_1 + X_2 + \cdots + X_n}{n}
      = \frac{1}{n}\sum_{i}^{n} X_i$$


Это случайный текст, который цитирует статью \citet{Tir} и статью \citet{Ak}.

\newpage
\bibliography{myrefs}{}
\bibliographystyle{chicago}


\end{document} 