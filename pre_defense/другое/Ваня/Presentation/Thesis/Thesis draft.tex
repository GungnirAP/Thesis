\documentclass[letterpaper,11pt]{article}
\usepackage{tabularx} % extra features for tabular environment
\usepackage{amsmath}  % improve math presentation
\usepackage{amssymb}
\usepackage{graphicx} % takes care of graphic including machinery
\usepackage{multirow}
\usepackage{bbding}
\usepackage{caption}
\usepackage{subcaption}
\usepackage{float}
\usepackage{multicol}
\usepackage{setspace}
\usepackage{booktabs}
\usepackage{adjustbox}
\usepackage{realhats}
\usepackage{array}
\usepackage{bbold}
\renewcommand\arraystretch{1.5}
\def\sym#1{\ifmmode^{#1}\else\(^{#1}\)\fi}
\doublespacing
\usepackage[margin=1in,letterpaper]{geometry} % decreases margins
\usepackage{cite} % takes care of citations
\usepackage[final]{hyperref} % adds hyper links inside the generated pdf file
\usepackage{color,hyperref}
\definecolor{darkblue}{rgb}{0.0,0.0,0.4}
\definecolor{darkgreen}{rgb}{0.0,0.3,0.0}
\definecolor{darkred}{rgb}{0.7,0.0,0.0}
\hypersetup{
	colorlinks=true,       % false: boxed links; true: colored links
	linkcolor=darkgreen,        % color of internal links
	citecolor=darkred,        % color of links to bibliography
	filecolor=magenta,     % color of file links
	urlcolor=darkblue         
}
\usepackage{blindtext}
\usepackage{bbm}
    \usepackage{enumitem} 
            \setlist{nosep}
                \usepackage{geometry} 
    \geometry{top=20mm}
    \geometry{bottom=20mm}
    \geometry{left=18mm}
    \geometry{right=18mm}
    
    %Drawing
    \usepackage{physics}
    \usepackage{amsmath}
    \usepackage{tikz}
    \usepackage{mathdots}
    \usepackage{yhmath}
    \usepackage{cancel}
    \usepackage{color}
    \usepackage{siunitx}
    \usepackage{array}
    \usepackage{multirow}
    \usepackage{amssymb}
    \usepackage{gensymb}
    \usepackage{tabularx}
    \usepackage{extarrows}
    \usepackage{booktabs}
    \usetikzlibrary{fadings}
    \usetikzlibrary{patterns}
    \usetikzlibrary{shadows.blur}
    \usetikzlibrary{shapes}
   
   \usepackage{amsthm}
   \usepackage[english]{babel}
   \newtheorem{definition}{Definition}
   \newtheorem{theorem}{Theorem}
   \newtheorem{proposition}{Proposition}
   \newtheorem{prediction}{Prediction}
   
   
   \usepackage[natbibapa]{apacite}
   
%++++++++++++++++++++++++++++++++++++++++


\begin{document}

\title{\textsc{Public good and attitude towards ethnic diversity} \\ \vspace{.5cm} \large \textsc{Thesis}}
\author{Ivan Dedyukhin}
\date{\today}
\maketitle

\begin{abstract}
 This paper contributes to the topic of relationship between public good provision and ethnic heterogeneity. It shows that attitude towards nationalities influence public good provision.    
\end{abstract}

\newpage

\tableofcontents

\newpage

\section{Introduction}

Does ethnic diversity affect the provision of public goods, or is it only the attitude towards nationalities that matters?

The answer to this question may have a beneficial influence on public policy towards ethnicities. This question is significant in Russia, mainly because it is multinational. In 2010 the index of ethnolinguistic diversity varied from 0.095 to 0.837 for Russian regions (\cite{Russia}). If the attitude towards heterogeneity matters, the state can start a policy of reconciliation, stop nationalism. If diversity matters, the government may not affect the regional division or the settlement of nationalities. Every answer to the stated research question is interesting for economists and politicians because of its effect on population welfare. 

\section{Literature review}

Ethnic diversity is an essential determinant of economic performance. For example, \cite{Africa} shows that ethnic diversity explains the difference in public policies between countries in Sub-Saharan Africa. These varieties explain lower economic growth in countries with higher ethnic fragmentation. 

Talking about public goods, the classical paper from \cite{AlesinaDivision} argues that the higher ethnic diversity is associated with the lower provision of some public goods. They presented a model in which different ethnic groups have different preferences for the public good to explain such dependence. They have found that spending on education, roads, sewers, and trash pickup is inversely related to fractionalization index, while police, health, and welfare had no effect. The endogeneity concern occurs here because the data was from US municipalities, and people may move between them. The paper from \cite{Kenya} deals with it because the ethnic patterns are stable and historically determined. They showed that ethnic fractionalization is associated with lower spending on primary schools. In addition, they presented a model based on the other mechanism of social sanctions. Higher ethnic fractionalization causes a failure of collective actions due to incapacity to implement sanctions. In addition, \cite{Indonesia} support all results of previous papers. The authors use the variation in ethnic fragmentation because of new jurisdictions and show that greater ethnic fractionalization causes increased deforestation. This is consistent with the paper on under provision of public goods. Moreover, the \cite{transfer} have found the same effect but presented a different mechanism. They argue that higher ethnic diversity increases the cost of producing public goods. It can be explained by the greater transaction costs and strengthened the problem of public action.

\cite{WhyUndermine} contributed to the mechanism of collective action. They ran several experiments and have found that the strategy selection mechanism is the main one. In similar conditions, agents from the same ethnic group play cooperative equilibrium, but agents from different groups do not. At the same time, they argue that the preferences mechanism is not the same one. However, \cite{Cooperation} have different results. They argue that evolutionary game theory predicts deviation from cooperative strategy, but social diversity helps deal with this problem. To prove that, the authors introduced an experiment with a public good game and showed that diversity promotes cooperation.

\cite{difference} argues that previous papers focused on fractionalization while the distance in social and economic characteristics also matters. They have found that economic inequality strengthens the effect shown in previous papers.

At this moment, it is crucial to elaborate on different polarization and fractionalization indexes. Firstly, \cite{Reynal-Querol} introduced polarization index instead of fractionalization. They argue that previous papers have shown a weak connection between fractionalization and conflict that explains a lower economic growth rate. The authors develop a new polarization index and show that it is better to explain the frequency of civil wars. Secondly, \cite{EstebanRay} introduced their theory of the measurement of polarization. They presented the polarization index and used it for analysis of ethnic diversity. \cite{EstebanRayConflict} distinguished between the occurrence of the conflict and its seriousness. They have shown that fractionalization and polarization tend to have opposite effects on both aspects, and the second one has a positive impact on the intensity of the conflict. These facts suggest that attitude toward ethnic diversity has a higher impact than its fragmentation. Other paper from \cite{alpha} presents different the family of polarization indices that I will use in the paper.

Talking about an attitude towards diversity, \cite{Foreign} explain determinants of attitudes towards foreigners and shows significant variation in the period from 1988 to 2008. \cite{Refuges} show that greater ethnic fractionalization and polarization are associated with lower support of refugees in Europe, which means a worse attitude to diversity. At the same time, this effect is not robust to different measures of ethnic diversity. 

Literature leaves me with the idea that attitude towards ethnic diversity affects public goods provision instead of actual fractionalization.

\section{Model}

In this section, I will describe the baseline model. The economy consists of agents and ethnic groups. Each agent $i$ belongs to some ethnicity $I$, and this affiliation is exogenous. Different agents are endowed with varying levels of income. The set of nationalities $E$ is fixed. Firstly, I will follow \cite{BBV} model of private provision of public goods. Secondly, I will take a case of state provision of a public good, where government is assumed to maximize the social welfare.

\subsection{Setup}

The setup almost repeats the baseline theoretical model from \cite{BreakUp}. The population of economy is $N$, and $p(I)$ is a population of ethnic group $I$. Hence
\[
    \sum_{I}p(I) = N
\]

I will define $\pi(I)$ as a share of ethnicity $I$  in total population of economy:
\[ \pi(I) = \frac{p(I)}{N} \]

Each agent's utility $u(x, g, H)$ depends on private consumption ($x$), public goods consumption (g), and ethnic heterogeneity ($H$). The utility function is twice continuously differentiable, strictly concave and increasing in $x$ and $g$. For simplicity, the prices of private and public goods are set to 1. I assume that ethnic heterogeneity reduces the utility an agent derives from the consumption of the public good $g$. It happens because the consumption of public good requires communication with other people. An agent may not be satisfied because of interactions with other nationalities that he does not like. This is the main mechanism that I will discuss further. Also, I refuse the claim of the \cite{AlesinaDivision} that different ethnic groups might have different preferences over private and public consumption because \cite{WhyUndermine} refutes this mechanism. That is why the utility function is the same for all nationalities with given heterogeneity. 

\subsection{First best}

I will take the quasi-linear utility function with convex preferences. This function meets all the requirements described in the baseline. The utility of each agent $i$ from a group $I$ in country $C$ is 
\begin{equation}
\label{eq:utility}
U_i = x_i + \gamma((1 - H_i) g)^\beta    
\end{equation}
where $x_i$ is a private consumption of an agent $i$, $H_i$ is ethnic diversity that she faces, $g$ is an amount of public good. I will define $H(I)$ as a heterogeneity of each agent in group $I$. $\beta \in (0,1) $ implies convexity of preferences, and $\gamma$ implies the degree of preference for the public good in comparison with the private.

Let me suppose that the total income is W. The Samuelson condition (\cite{Samuelson}) gives us the first best allocation of resources.
\begin{equation}
    \label{eq:SC}
    \sum_{i = 1}^{N} MRS_i = 1
\end{equation}

\[ \sum_{I} p(I) \gamma(1-H(I))^\beta \beta g^{\beta - 1} = 1 \]

Hence, the Equation \ref{eq:FB} gives us the first best amount of public good

\begin{equation}
   g^{FB} = \left( \gamma\beta \sum_{I} p(I) (1-H(I))^{\beta} \right) ^\frac{1}{1 - \beta } 
   \label{eq:FB}
\end{equation}

\subsection{Private provision}

Let me suppose that public goods are specific to each country and are financed through private provision. 
The budget constraint for each agent is following
\[x_i(I) + g_i(I) \le  w_i(I) \]
where $w_i(I)$ is an income of agent from group $I$, $x_i(I)$ is private consumption of agent from group $I$,  and $g_i(I)$ is a contribution of an agent from group $I$. Utility function is increasing in private provision, so I claim that budget constraint is binding.

So, the utility function is
\[ U = w_i(I) - g_i(I) + \gamma( (1 - H(I) ) g)^\beta \] 
where $g$ is a total provision of public good.

The optimization problem for each agent is following:

\[ \max_{g_i > 0} w_i(I) - g_i(I) + \gamma( (1 - H(I)^\delta) (g_{-i} + g_i))^\beta  \]
where $g_{-i}$ is a total contribution of all other agents.
The solution of this problem leads to the following proposition.

\begin{proposition}
The amount of public good in the economy with its private provision is decreasing with the increase of ethnic heterogeneity for a group with lowest one.
\end{proposition}

\begin{proof}
The FOC is
\[ -1 + \beta\gamma (1 - H(I))^{\beta} ((g_i(I) + g_{-i} ))^{\beta-1} = 0 \]

\[ g_i(I) = \max \left\{\frac{1}{(\beta\gamma )^\frac{1}{\beta - 1} (1 - H(I))^\frac{\beta}{\beta-1} } - g_{-i} ; 0  \right\}   \]

We may notice that groups differ only heterogeneity and higher diversity cause greater bliss amount. The wanted amount of a group $I$ with heterogeneity $H$ is 

\[ G(I) =  \frac{1}{(\beta\gamma )^\frac{1}{\beta - 1} (1 - H(I))^\frac{\beta}{\beta-1} } \]

The Nash equilibrium amount is bliss amount for group with smallest heterogeneity. This group will contribute and all others groups will contribute zero because their bliss amount is less and their contribution will lead to increase of public good.

Hence 

\[g = \frac{1}{(\beta\gamma )^\frac{1}{\beta - 1} (1 - \min_{I} H(I))^\frac{\beta}{\beta-1} } \]

This amount may be greater than the total income of the group with the lowest heterogeneity. Let me denote $W(I)$ as total income of a group $I$ than 

\begin{equation}
    g^{PP} = \min \left\{ (\beta\gamma  ( 1 - \min_{I} H(I)))^\frac{\beta}{1 -\beta}; W( \text{argmin}_{I} H(I)) \right\}
    \label{eq:privat}
\end{equation}

where $g^{PP}$ is an equilibrium amount for the economy with private provision of public good.

Higher ethnic heterogeneity for the group that faces the lowest heterogeneity decreases the total provision of public good 
\end{proof}

\subsubsection{Alienation effect}

For theoretical purposes, the measure of ethnic diversity will be the weighted attitudes distance between that resident and all other residents of economy. This corresponds to the expected distance between an agent from a region $I$ and a randomly drawn agent from economy. The greater distance means worse attitude and greater heterogeneity.

\[ H(I) = \sum_{J} \frac{p(J) \alpha(I, J)}{N} = \sum_{J} \pi(J) \alpha(I,J) \]
where $\alpha(I, J)$ is an attitude of group $I$ towards the group $J$ (distance between groups). Important to say that $\alpha(I, J) \ne \alpha(J, I)$. Let me suppose that the identification $\alpha(I,I) = 0$ for all $I$ for all $C$. I will call $alpha(I, J)$ alienation if $I \ne J$. I chose that form because the attitude ($\alpha$) and probability to meet ($\frac{p(J)}{p(C)}$) matters the satisfaction of public good.

Now suppose two groups: ethnic minority(1) and ethnic majority(2). Majority and minority means that $\pi(1) < \pi(2)$ Alienation is $\alpha(1,2) = \alpha_1$; $\alpha(2,1) = \alpha_2$, and identification is $\alpha(1,1) = \alpha(2,2) = 0$. Hence, we may say that
\[ H(1) = \sum_{J} \frac{p(J) \alpha(1, J)}{N} = \pi(2) * \alpha_1 \]
\[ H(2) = \sum_{J} \frac{p(J) \alpha(2, J)}{N} = \pi(1) * \alpha_2 \]
Also for simplicity let me define $\pi_1 = \pi(1)$ and $\pi_2 = \pi(2)$

\begin{proposition}
    Increase in alienation of any group causes a decrease or immutability in public good provision.
\end{proposition}

\begin{proof}
From, the previous results and equation \ref{eq:privat} we can notice that

\[ g^{PP} = \min \left\{ (\beta\gamma  ( 1 - \min_{I} H(I)))^\frac{\beta}{1 -\beta}; W( \text{argmin}_{I} H(I)) \right\} \]

Hence we had two cases:
\begin{enumerate}
    \item $H(1) > H(2)$:
    \[ g^{PP} = \min \left\{ (\beta\gamma  ( 1 - H(2)))^\frac{\beta}{1 -\beta}; W(2) \right\} \]
    The heterogeneity does not influence the income, so let me see the the internal solution.
    \[ g^{PP} (\beta\gamma  ( 1 - H(2)))^\frac{\beta}{1 -\beta} \]
    \[ g^{PP} (\beta\gamma  ( 1 - \pi_1 * \alpha_2))^\frac{\beta}{1 -\beta} \]
    Greater alienation of minor group causes decreases the amount of public good, while alienation of major group does not
    
    \item $H(2) \ge H(1)$:
    \[ g^{PP} = \min \left\{ (\beta\gamma  ( 1 - H(1)))^\frac{\beta}{1 -\beta}; W(1) \right\} \]
    The heterogeneity does not influence the income, so let me see the the internal solution.
    \[ g^{PP} (\beta\gamma  ( 1 - H(1)))^\frac{\beta}{1 -\beta} \]
    \[ g^{PP} (\beta\gamma  ( 1 - \pi_2 * \alpha_1))^\frac{\beta}{1 -\beta} \]
    Greater alienation of major group causes decreases the amount of public good, while alienation of minor group does not
    
    \item Increase in $\alpha$ cause a change of group:
    
    In this case the minimal heterogeneity will not become lower because higher alienation does not cause decrease in heterogeneity. Hence, greater alienation will cause decrease or immutability of public good provision.
    
\end{enumerate}

\end{proof}

This result is logical because it increases the heterogeneity of one group without decreasing the heterogeneity for others. So, this is a consequence from proposition 1.

%%%%%%%%%%%%%%%%%%%%%%%%%%%%%%%%%%%%%%%%%%%%%%%%%%%%%%%

Here we supported results of previous papers on connection between public good provision and ethnic heterogeneity. The next proposition comes from the last equation. 

\begin{proposition}
    Fractionalization and the alienation are complements.
\end{proposition}

\begin{proof}
    I will work with the numerical case here.
    
    \[ g =  - \pi_1\alpha_1 + \pi_1^2\alpha_1 + 1 -  \pi_1 \alpha_2 + \pi_1^2 \alpha_2  \]
    \[\frac{\partial g}{ \partial \pi_1} =  (2\pi_1 - 1)(\alpha_1 + \alpha_2)  \]
    \[\frac{\partial^2 g}{ \partial \pi_1 \partial \alpha_1} =  2(\alpha_1 + \alpha_2) > 0 \]
    
    The mixed derivative is greater than zero, so they are complements.
\end{proof}

The model shows that attitudes and fractionalization are complements in such preferences setting which assumption recognized by many scientists \textbf{references to them?} 

\subsubsection{Alienation and shares}

Here I will suppose that attitude towards other group is influenced by its share. For example, \cite{Refuges} argues that higher amount of refuges in European countries causes the worse attitude towards them. \textbf{Esteban, Ray. Polarization}.

\textbf{Funcrional form}

Hence, let me suppose that in the setting with two ethnic groups

\[ \alpha(1,2) = \lambda \pi_2 \Rightarrows H(1,C) = \lambda \pi_2^2 \]
\[ \alpha(2,1) = \lambda \pi_1 \Rightarrows H(1,C) = \lambda \pi_1^2  \]
\begin{proposition}
    The worse attitude towards diversity decreases the amount of public good.
\end{proposition}

\begin{proof}

\[ g = \pi_1 (\beta\gamma )^\frac{1}{1 - \beta} (1 - (\lambda \pi_2^2)^\delta)^\frac{\beta}{1 - \beta}   + \pi_2 (\beta\gamma )^\frac{1}{1 - \beta} (1 - (\lambda \pi_1^2)^\delta)^\frac{\beta}{1 - \beta}  \]

Let me analyse the simple numerical case ($\delta = 1$ and $\beta = 0,5$, $\gamma = 2$). Hence:

\[ g = \pi_1 (1 - \lambda \pi_2^2) + \pi_2 (1 - \lambda \pi_1^2) \]
\[ g = \pi_1 (1 - \lambda (1 - \pi_1)^2) + (1 - \pi_1) (1 - \lambda \pi_1^2) \]
\[g = 1 + \lambda \pi_1^2 - \lambda \pi_1 \]
\[ \frac{\partial g}{\partial \pi_1} = \lambda (2\pi_1 - 1) < 0 \]

Since the group 1 is minor ($\pi_1 < \frac{1}{2}$), the derivative is negative. What is more interesting, and probably, obvious, greater sensitivity to share strengthen the effect.

\end{proof}

%%%%%%%%%%%%%%%%%%%%%%%%%%%%%%%%%%%%%%%%%%%%%

\subsection{State provision}

Let me assume that government is benevolent and maximizes the social welfare. The financing of public good goes through tax as a percent from income. The optimization problem for government is

\[ \max_{t\in[0,1], g\ge 0} \sum_{I} p(I) ( (1-t)w_i(I) + \gamma( (1 - H(I)) g)^\beta)  \]
\[ g \le t\sum_{I} p(I) w_i(I)  \]

\begin{proposition}
Higher heterogeneity for at least one ethnic group leads to lower provision of public good.
\end{proposition}

\begin{proof}

\[ g = t\sum_{I} p(I) w_i(I)  \]
because the social welfare is increasing in public good provision. 
Also, let me set $W = \sum_{I} p(I) w_i(I) $ as a total income. Hence $g = tW$, and
\[ \max_{t\in[0,1]} (1-t) W + \sum_{I} p(I) \gamma( (1 - H(I)) tW)^\beta)  \]
The FOC is 
\[ -W + \sum_{I} p(I) \gamma ((1 - H(I))^\beta W^\beta \beta t^{\beta - 1} = 0 \]

\[t = \min \left\{ \left( \gamma W^{\beta - 1} \beta\sum_{I} p(I) (1 - H(I))^\beta \right)^\frac{1}{1 - \beta}; 1 \right\} \]

\begin{equation}
    g^{SP} = \min \left\{ \left( \gamma  \beta\sum_{I} p(I) (1 - H(I))^\beta \right)^\frac{1}{1 - \beta}; W \right\}
    \label{eq:public}
\end{equation}

The provision of public good is decreasing in $H(I,C)$
\end{proof}

In addition, the Equation \ref{eq:public} gives us the amount of public good in the situation of state provision.

\subsubsection{Alienation effect}

Let me take the same heterogeneity form as in the case of private provision.

\begin{proposition}
    Increase in alienation of any group causes a decrease in public good provision.
\end{proposition}

\begin{proof}
    From equation \ref{eq:public} I have 
    \[ g^{SP} = \min \left\{ \left( \gamma  \beta\sum_{I} p(I) (1 - H(I))^\beta \right)^\frac{1}{1 - \beta}; W \right\} \]
    The heterogeneity does not affect income, so let me elaborate on the inner solution.
    \[ g^{SP} = \left( \gamma  \beta \left( p(1) (1 - \pi_2\alpha_1)^\beta + p(2) (1 - \pi_1\alpha_2)^\beta  \right) \right)^\frac{1}{1 - \beta} \]
    Increase in any $\alpha$ cause a decrease of the amount of public good.
\end{proof}

The theoretical analysis of different cases of economy with a private or state provision of public good gives the prediction that I will check empirically.

\begin{prediction}
    Alienation have negative effect on the amount of public good provided.
\end{prediction}

%%%%%%%%%%%%%%%%%%%%%%%%%%%%%%%%%%%%%%%%%%%%%%%%%%%

\subsubsection{ Alienation and shares}

These results lead to a following prediction

\begin{prediction}
    Ethnic fractionalization and alienation have negative and complement effect on the state provision of public good.
\end{prediction}

%%%%%%%%%%%%%%%%%%%%%%%%%%%%%%%%%%%%%%%
\subsection{Comparison}

In this section I want to compare the amount of public good that I got in equations \ref{eq:FB}, \ref{eq:privat}. and \ref{eq:public}. Hence, I want to compare first best, private provision, and state provision, respectively. Let me for simplicity assume that income is as great that we can neglect it. In this case amounts are following:

\[  g^{FB} = \left( \gamma\beta \sum_{I} p(I) (1-H(I))^{\beta} \right) ^\frac{1}{1 - \beta }  \]

\[ g^{PP} =  (\beta\gamma  ( 1 - \min_{I} H(I)))^\frac{\beta}{1 -\beta} \]

\[  g^{SP} = \left( \gamma  \beta\sum_{I} p(I) (1 - H(I))^\beta \right)^\frac{1}{1 - \beta}\]

We can notice that state provision of public goods leads to first best amount of public goods but it does not mean that it leads to first best of allocation of contributions to the public good and its efficiency.

\begin{proposition}
If $N < (\beta\gamma)^{\beta-1}$ than the public good amount in the case of state provision is lower than in the case of private provision
\end{proposition}

\begin{proof}
    Let me compare in the case of equal heterogeneity $H$ that all groups faces

    \[ g^{PP} =  (\beta\gamma  ( 1 - H)^\frac{\beta}{1 -\beta}\;\;\; \bigvee \;\;\; g^{SP} = \left( \gamma  \beta\sum_{I} p(I) (1 - H)^\beta \right)^\frac{1}{1 - \beta} \]

    \[  (\beta\gamma)^{\beta-1}  (1 - H)^\beta \;\;\; \bigvee \;\;\;  N (1 - H)^\beta \]

    \[  (\beta\gamma)^{\beta-1} \;\;\; \bigvee \;\;\;  N \]

    Hence, if $N < (\beta\gamma)^{\beta-1}$ than the amount of public good with state provision is greater than in situation with private provision. The case of different heterogeneity may be interpreted as increase of some $H(I)$ comparing to the considered case. So, the amount of state provision decreases while the amount of private provision stays same.
\end{proof}

The Figure shows the shows the area when state provision public good amount is greater for different parameters. It turns out that we can say that greater population means that more likely the state provision amount will be greater than the private provision amount. At the same time, if agents stronger prefer public good (increase in $\gamma$) then more likely the state provision amount will be greater.

\begin{figure}[h]
    \centering
    \includegraphics[scale = 0.4]{Thesis/Econometrics US/Parameters.png}
    \caption{Areas when state provision public good amount is greater}
    \floatfoot{The abscissa axis is $\beta$; the ordinate axis is $N$; the blue area is for $\gamma = 5$; the red area is for $\gamma = 1$}
    \label{fig:param}
\end{figure}


\section{Empirical evidence}

I test my predictions with a panel dataset of
public goods spending, ethnic diversity and attitudes towards it in 97  U.S. cities for the period 2006 - 2017.

\subsection{Data and sources}

I use the ethnic fractionalization (FRAC), Reynal-Querol polarization (RQ) indices as a measure of ethnic heterogeneity. FRAC measures the probability that two randomly drawn people from a city belonf to different ethnic groups. I will consider the population distribution by race, so I construct FRAC as follows:
\begin{equation}
    FRAC = \sum_{i} \pi_i (1 - \pi_i) = 1 - \sum_{i}\pi_i^2
\end{equation}
where $pi_i$ is a share of race $i$ in cities population and
\[ i \in \{\text{White, Black, Asian and Pacific Islander, American Indian, Other} \} \]

I will not use other polarization like Reynal-Querol which is trying to capture intuitively how far the distribution of the groups is from a bipolar distribution because it captures not the fact of the heterogeneity and also the potential for conflict. However, these indices are highly correlated in collected dataset. This fact is shown in the Figure \ref{fig:rq-frac}, which repeats the one from \cite{Reynal-Querol}. I constructed RQ as follows:
\begin{equation}
    RQ=1-\sum_{i=1}^{N}\left[\frac{0.5-\pi_{i}}{0.5}\right]^{2} \pi_{i}
\end{equation}

\begin{figure}[h]
    \centering
    \includegraphics[scale = 0.4]{Thesis/Econometrics US/rq - frac.png}
    \caption{Corellation between Reqnal-Querol and fractionalization indices}
    \label{fig:rq-frac}
\end{figure}

Other famous possibility to capture polarization is Esteban-Ray (ER) (\cite{EstebanRay}) which can be calculated as follows:
\[ER = \sum_{i} \sum_{j} d_{ij} \pi_i \pi_j  \]
where $d_{ij}$ is a distance between groups. The inability to calculate these distances is the main problem of using index. The pure fractionalization is the best index to measure the fact of heterogeneity without other concerns as attitudes.

I follow the racial classification used by the U. S. Census. The source of the data is \hyperlink{https://usa.ipums.org/usa/}{IPUMS USA}  project that collects, preserves and harmonizes U.S. census microdata. I had data on a random drawn people from U.S. counties, so my identification strategy has first assumption that balancing property holds for races in city level. This database was also used for constructing some controls as mean households income and inequality calculated by dividing average income by median one.

Note that there is no "Hispanic" option in racial classifications in the Census. However, there is a correlation (0.54) between "Hispanic" and "Other"in the above classification. Many Hispanics apparently respond "Other" because they do not feel accurately represented in the multiple racial choice provided by the Census.

The attitude towards diversity index (ALIENATION) is created by myself. I calculate the share of co-racial marriages in total number of marriages in a city. I assume that the greater fraction cause less tolerance to diversity for city's population. The possible caveat is that this INDEX is dependent on fractionalization because higher ethnic heterogeneity increase the probability of co-racial marriage.
The Figure \ref{fig:frac-index} shows significant variation in attitudes towards diversity for same ethnic fractionalization. Therefore, I argue that marriage represents alienation rather than ethnic heterogeneity.

\begin{figure}
    \centering
    \includegraphics[scale = 0.4]{Thesis/Econometrics US/frac - index.png}
    \caption{Scatter plot for attitude index and ethnic fractionalization}
    \label{fig:frac-index}
\end{figure}

Fiscal variable are taken from The Fiscally Standardized Cities (FiSC) database from Lincoln Institute of Land Policy. This database makes possible to compare cities finances because public goods and spending are financed by different level of governments. The data are available for 212 U.S. cities for the 1977–2017 period on the Lincoln Institute of Land Policy’s \hyperlink{https://www.lincolninst.edu/research-data/data-toolkits/fiscally-standardized-cities}{website}. I have taken public spending variable by goals, city population and taxes collected. All fiscal variables are calculated per capita in 2017 USD.

Fiscal and IPUMS datasets were merged by the year and city name. Descriptive statistics is presented in the Table \ref{tab:desc}. We can notice the significant variation in all variables of interest: ethnic fractionalization or polarization and attitude towards diversity. The variation in fiscal variables is also huge, but we can notice the absence if spending in health or libraries in some cities. It may happen because of the absence of public hospitals,libraries of sewerage facilities in small cities. Therefore, I can not drop thes observation because it may take place and zero spending is also a decision.

\begin{table}[h]
    \centering
    \footnotesize
    {
\def\sym#1{\ifmmode^{#1}\else\(^{#1}\)\fi}
\begin{tabular}{l*{1}{cccc}}
\hline\hline
                    &\multicolumn{4}{c}{(1)}                            \\
                    &\multicolumn{4}{c}{}                               \\
                    &        Mean&          S.D.&         Min&         Max\\
\hline
City Population     &    557022.8&     1060585&       80215&     8475976\\
Log of city population&    12.68124&    .8683314&    11.29247&    15.95275\\
Taxes collected     &    2198.987&    1221.208&      732.71&    11223.87\\
General expenditures&    5952.036&    2291.052&     2033.46&    21446.74\\
Secondary education expenditures&    2007.007&    644.0921&      544.84&     4811.56\\
Libraries expenditures&    51.92246&    34.49064&           0&      585.74\\
Public welfare expenditures&    268.4206&    604.8267&           0&     5725.41\\
Hospital expenditures&    282.1485&     582.429&           0&     4432.26\\
Health expenditures &    217.3635&    228.5996&           0&     2198.96\\
Highways expenditures&    232.4785&    133.0477&        6.21&     1067.75\\
Public safety expenditures&    773.3932&    265.9719&      268.43&     2312.49\\
Sewerage expenditures&    245.4629&    160.8913&           0&     1070.79\\
Administration expenditures&    308.1253&    168.1048&       56.07&     1698.06\\
Parks and recreation expenditures&    178.9966&    136.8253&        1.03&     1410.71\\
Ethnic fractionalization&    .4420603&    .1377596&    .0853811&    .6934802\\
Ethnic fractionalization among married&    .4580479&    .1321692&    .0917879&    .6981086\\
Reynal-Querol polarization&    .6988702&    .1972032&     .140374&    .9807938\\
Alienation          &     .189907&    .0825309&    .0536585&    .4158192\\
Mean HH income      &    634356.8&    404314.1&    88548.67&     2851736\\
Inequality          &    11.32726&    7.060934&    1.225278&     47.1012\\
\hline
Observations        &         900&            &            &            \\
\hline\hline
\end{tabular}
}

    \caption{Descriptive statistic for US cities 2006 - 2017}
    \label{tab:desc}
\end{table}

\subsection{Empirical estimations}

The Basic specification is 

\begin{equation}
\label{eq: basic}
    Fiscal_i = \alpha + \beta_1 \times FRAC_i + \beta_2 \times INDEX_i + \beta_3 \times INDEX_i \times FRAC_i + \gamma X_i
\end{equation}
where $X_i$ are covariates.

Previous papers have shown that $\beta_1$ is negative. My theoretical predictions and hypothesis expects $\beta_2$ and $\beta_3$ to be negative. The coefficient $\beta_2$ is responsible for the effect of attitude itself, and $\beta_3$ is responsible for complementarity. The subsection \ref{subsec: alienation} explores the effect of attitude towards heterogeneity while subsection \ref{subsec: complement} shows complementarity of fractionalization and index.

\begin{table}[H]
    \centering
    \footnotesize
    {
\def\sym#1{\ifmmode^{#1}\else\(^{#1}\)\fi}
\begin{tabular}{l*{6}{c}}
\hline\hline
                    &\multicolumn{1}{c}{(1)}&\multicolumn{1}{c}{(2)}&\multicolumn{1}{c}{(3)}&\multicolumn{1}{c}{(4)}&\multicolumn{1}{c}{(5)}&\multicolumn{1}{c}{(6)}\\
                    &\multicolumn{1}{c}{Secondary education}&\multicolumn{1}{c}{Social services}&\multicolumn{1}{c}{Libraries}&\multicolumn{1}{c}{Parks}&\multicolumn{1}{c}{Police}&\multicolumn{1}{c}{Welfare}\\
\hline
Ethnic fractionalization&     -0.0317         &    -0.00455         &     -0.0104         &    0.000741         &     -0.0141         &    -0.00247         \\
                    &    (0.0555)         &    (0.0427)         &   (0.00721)         &    (0.0286)         &    (0.0303)         &    (0.0200)         \\
[1em]
Alienation          &      0.0938         &    -0.00811         &     -0.0171         &     -0.0431         &     -0.0332         &    -0.00416         \\
                    &     (0.116)         &    (0.0892)         &    (0.0151)         &    (0.0597)         &    (0.0632)         &    (0.0419)         \\
[1em]
Index x Frac        &      0.0141         &      0.0376         &      0.0450         &      0.0110         &      0.0557         &      0.0283         \\
                    &     (0.239)         &     (0.184)         &    (0.0311)         &     (0.123)         &     (0.131)         &    (0.0864)         \\
[1em]
Mean HH income      &    3.92e-08\sym{***}&   -2.94e-08\sym{***}&    1.08e-09         &   -9.99e-09         &   -1.04e-08         &    3.33e-09         \\
                    &  (1.14e-08)         &  (8.81e-09)         &  (1.49e-09)         &  (5.89e-09)         &  (6.24e-09)         &  (4.13e-09)         \\
[1em]
Inequality          &    -0.00188\sym{**} &     0.00148\sym{**} &   0.0000261         &    0.000489         &    0.000510         &   -0.000354         \\
                    &  (0.000728)         &  (0.000561)         & (0.0000946)         &  (0.000375)         &  (0.000397)         &  (0.000263)         \\
[1em]
Log of city population&       0.145\sym{***}&      0.0387         &    -0.00155         &     -0.0393\sym{*}  &     -0.0140         &      0.0196         \\
                    &    (0.0326)         &    (0.0251)         &   (0.00423)         &    (0.0168)         &    (0.0178)         &    (0.0118)         \\
[1em]
Constant            &      -1.476\sym{***}&      -0.383         &      0.0326         &       0.536\sym{*}  &       0.314         &      -0.211         \\
                    &     (0.415)         &     (0.319)         &    (0.0539)         &     (0.214)         &     (0.226)         &     (0.150)         \\
\hline
Observations        &         900         &         900         &         900         &         900         &         900         &         900         \\
\hline\hline
\multicolumn{7}{l}{\footnotesize Standard errors in parentheses}\\
\multicolumn{7}{l}{\footnotesize \sym{*} \(p<0.05\), \sym{**} \(p<0.01\), \sym{***} \(p<0.001\)}\\
\end{tabular}
}
    \caption{OLS general regression}
    \label{tab:basic}
\end{table}

The results of the specification are presented in Table \ref{tab:basic}. It shows no significance due to possible multicolinearity that is shown by \cite{Refuges}. The authors argued that greater amount of refuges in European countries causes worse attitude towards them. The proof of multicollinearity is presented in Table \ref{tab:frac - index}. The results show that higher ethnic fractionalization cause worse attitudes towards diversity.

\begin{table}[H]
    \centering
    {
\def\sym#1{\ifmmode^{#1}\else\(^{#1}\)\fi}
\begin{tabular}{l*{6}{c}}
\hline\hline
                    &\multicolumn{1}{c}{(1)}&\multicolumn{1}{c}{(2)}&\multicolumn{1}{c}{(3)}&\multicolumn{1}{c}{(4)}&\multicolumn{1}{c}{(5)}&\multicolumn{1}{c}{(6)}\\
                    &\multicolumn{1}{c}{Alienation}&\multicolumn{1}{c}{Alienation}&\multicolumn{1}{c}{Alienation}&\multicolumn{1}{c}{Alienation}&\multicolumn{1}{c}{Alienation}&\multicolumn{1}{c}{Alienation}\\
\hline
Ethnic fractionalization&       0.272\sym{***}&       0.136\sym{***}&       0.124\sym{***}&       0.290\sym{***}&       0.137\sym{***}&       0.117\sym{***}\\
                    &    (0.0490)         &    (0.0343)         &    (0.0319)         &    (0.0483)         &    (0.0329)         &    (0.0320)         \\
                    &                     &                     &   (0.00412)         &                     &                     &   (0.00511)         \\
[1em]
Mean HH income      &                     &                     &                     &    6.36e-08         &    1.90e-08         &   -9.28e-09         \\
                    &                     &                     &                     &  (3.61e-08)         &  (1.16e-08)         &  (1.12e-08)         \\
[1em]
Inequality          &                     &                     &                     &    -0.00545\sym{*}  &   -0.000182         &    0.000454         \\
                    &                     &                     &                     &   (0.00253)         &  (0.000683)         &  (0.000770)         \\
[1em]
Log of city population&                     &                     &                     &     0.00181         &      0.0333         &     -0.0270         \\
                    &                     &                     &                     &   (0.00913)         &    (0.0290)         &    (0.0336)         \\
[1em]
Constant            &      0.0698\sym{***}&       0.130\sym{***}&       0.128\sym{***}&      0.0600         &      -0.303         &       0.473         \\
                    &    (0.0191)         &    (0.0152)         &    (0.0146)         &     (0.117)         &     (0.370)         &     (0.427)         \\
\hline
Observations        &         900         &         900         &         900         &         900         &         900         &         900         \\
\hline\hline
\multicolumn{7}{l}{\footnotesize Standard errors in parentheses}\\
\multicolumn{7}{l}{\footnotesize \sym{*} \(p<0.05\), \sym{**} \(p<0.01\), \sym{***} \(p<0.001\)}\\
\end{tabular}
}

    \caption{Relationship between attitude and heterogeneity.}
    \label{tab:frac - index}
\end{table}

\subsubsection{Alienation effect}
\label{subsec: alienation}

Firstly, I want to identify effect of the attitudes towards diversity. Simple OLS model is not applicable here because of multicollinearity of attitude index and ethnic fractionalization. 

I will use propensity-score-matching and estimating dose-response function with alienation as a treatment. \cite{prop} define propensity function as the conditional density of the actual treatment given the observed covariates. Hence, weak unconfoundedness holds and \cite{prop} show that the GPS (generalised propensity score) can be used to eliminate any biases associated with differences in the covariates.

Firstly, using \cite{package} STATA package that help estimate does response function, I calculated propensity function with covariates fractionalization, polarization, mean income, inequality, taxes and city population.
\[ r(t, x)=f_{T \mid X}(t \mid x) \]
Secondly, I estimate the conditional expectation of
the outcome as a function of two scalar variables, the treatment level T and the GPS R:
\[ \beta(t, r)=E(Y \mid T=t, R=r) \]
The form is following:
\[ Fiscal_i = \alpha + \beta_1 \times INDEX_i + \beta_2 \times GPS_i + \beta_3 \times INDEX_i \times GPS_i \]
In the third step, I finally estimate the dose–response function,
\[ \mu(t)=E[\beta\{t, r(t, X)\}] \]
This method helps me to get fractionalization and polarization fixed and focus only on alienation.

Results are presented in Table \ref{tab:prop_score}. We can notice the expected and significant results only for secondary education, parks, and police fiscal dependent variable. Social services and public welfare are increased with greater alienation and it can possibly be caused by inability of collective action to help neighbor.

I have shown that the alienation and attitude towards diversity is essential determinant of the public goods provision.

\begin{table}[h]
    \footnotesize
    \centering
    {
\def\sym#1{\ifmmode^{#1}\else\(^{#1}\)\fi}
\begin{tabular}{l*{6}{c}}
\hline\hline
                    &\multicolumn{1}{c}{(1)}&\multicolumn{1}{c}{(2)}&\multicolumn{1}{c}{(3)}&\multicolumn{1}{c}{(4)}&\multicolumn{1}{c}{(5)}&\multicolumn{1}{c}{(6)}\\
                    &\multicolumn{1}{c}{Parks}&\multicolumn{1}{c}{Libraries}&\multicolumn{1}{c}{ Education}&\multicolumn{1}{c}{Police}&\multicolumn{1}{c}{Welfare}&\multicolumn{1}{c}{Social services}\\
\hline
Alienation          &     -0.0367\sym{*}  &     0.00555         &      -0.219\sym{**} &     -0.0710\sym{**} &       0.215\sym{***}&       0.175\sym{*}  \\
                    &    (0.0184)         &   (0.00448)         &    (0.0823)         &    (0.0262)         &    (0.0315)         &    (0.0749)         \\
[1em]
GPS&     0.00105         &     0.00114\sym{***}&    -0.00225         &    -0.00573\sym{**} &    0.000328         &    -0.00915         \\
                    &   (0.00128)         &  (0.000311)         &   (0.00571)         &   (0.00182)         &   (0.00219)         &   (0.00520)         \\
[1em]
Alienation \times GPS      &    -0.00401         &    -0.00479\sym{***}&      0.0244         &      0.0157\sym{*}  &    -0.00175         &      0.0208         \\
                    &   (0.00559)         &   (0.00136)         &    (0.0249)         &   (0.00793)         &   (0.00954)         &    (0.0227)         \\
[1em]
Constant            &      0.0369\sym{***}&     0.00685\sym{***}&       0.389\sym{***}&       0.159\sym{***}&    -0.00626         &      0.0980\sym{***}\\
                    &   (0.00500)         &   (0.00121)         &    (0.0223)         &   (0.00710)         &   (0.00854)         &    (0.0203)         \\
\hline
Observations        &         900         &         900         &         900         &         900         &         900         &         900         \\
\hline\hline
\multicolumn{7}{l}{\footnotesize Standard errors in parentheses}\\
\multicolumn{7}{l}{\footnotesize \sym{*} \(p<0.05\), \sym{**} \(p<0.01\), \sym{***} \(p<0.001\)}\\
\end{tabular}
}

    \caption{Propensity score matching results}
    \label{tab:prop_score}
\end{table}

These results are slightly consistent with results of general OLS modifications of specification \ref{eq: basic} in Tables \ref{tab: parks} - \ref{tab: social service} in Appendix. The only robust result in both strategies is parks and recreational spending. This is the expenditure part that is ideal fits my mechanism explanation. Park is a place where people meet each other. The greater alienation towards people in park will decrease the utility from consumption of this public good.

\subsubsection{Complementarity effect}
\label{subsec: complement}

Basic OLS regression from specification \ref{eq: basic} is presented in Table \ref{tab:basic}. So, we can not see significant effect because of high probability of multicolinearity as I said earlier. At the same time, we can see some significant complimentary effect evidence in Table \ref{tab: parks} in specifications (3) and (6) or Table \ref{tab: libraries} in specification (3), but this result is not robust. Hence, I decided to use the same propensity-score matching because I need the effect for given fractionalization and other covariates. The results are presented in Table \ref{tab:interaction}. I have found that the effect of alienation and actual diversity on public good provision is complementary for that one that requires social interactions like parks and libraries. This result is consistent with the suggested mechanism.

\begin{table}[H]
    \centering
    \footnotesize
    {
\def\sym#1{\ifmmode^{#1}\else\(^{#1}\)\fi}
\begin{tabular}{l*{6}{c}}
\hline\hline
                    &\multicolumn{1}{c}{(1)}&\multicolumn{1}{c}{(2)}&\multicolumn{1}{c}{(3)}&\multicolumn{1}{c}{(4)}&\multicolumn{1}{c}{(5)}&\multicolumn{1}{c}{(6)}\\
                    &\multicolumn{1}{c}{Parks}&\multicolumn{1}{c}{Libraries}&\multicolumn{1}{c}{Secondary education}&\multicolumn{1}{c}{Police}&\multicolumn{1}{c}{Welfare}&\multicolumn{1}{c}{Social services}\\
\hline
Index x Frac        &      -0.121\sym{**} &     -0.0192         &      -0.844\sym{***}&      -0.335\sym{***}&      0.0401         &       0.499\sym{**} \\
                    &    (0.0468)         &    (0.0114)         &     (0.210)         &    (0.0672)         &    (0.0815)         &     (0.192)         \\
[1em]
Alienation          &      0.0516\sym{*}  &      0.0128\sym{*}  &       0.159         &       0.138\sym{***}&       0.176\sym{***}&     -0.0889         \\
                    &    (0.0234)         &   (0.00572)         &     (0.105)         &    (0.0336)         &    (0.0407)         &    (0.0960)         \\
[1em]
GPS&    0.000546         &    0.000275\sym{*}  &    -0.00534\sym{*}  &    -0.00290\sym{***}&   -0.000115         &    -0.00240         \\
                    &  (0.000505)         &  (0.000123)         &   (0.00227)         &  (0.000726)         &  (0.000879)         &   (0.00207)         \\
[1em]
interaction         &    -0.00853\sym{*}  &    -0.00285\sym{**} &      0.0675\sym{***}&      0.0132\sym{*}  &     0.00405         &     0.00281         \\
                    &   (0.00368)         &  (0.000899)         &    (0.0165)         &   (0.00529)         &   (0.00641)         &    (0.0151)         \\
[1em]
Constant            &      0.0331\sym{***}&     0.00792\sym{***}&       0.401\sym{***}&       0.152\sym{***}&    -0.00381         &      0.0986\sym{***}\\
                    &   (0.00414)         &   (0.00101)         &    (0.0186)         &   (0.00595)         &   (0.00721)         &    (0.0170)         \\
\hline
Observations        &         900         &         900         &         900         &         900         &         900         &         900         \\
\hline\hline
\multicolumn{7}{l}{\footnotesize Standard errors in parentheses}\\
\multicolumn{7}{l}{\footnotesize \sym{*} \(p<0.05\), \sym{**} \(p<0.01\), \sym{***} \(p<0.001\)}\\
\end{tabular}
}

    \caption{Estimating the complementarity effect}
    \label{tab:interaction}
\end{table}


\section{Discussion}

This section will discuss some possible drawbacks of the bachelor thesis.

There are some extensions for the model. The first and the most obvious one is that transaction costs may occur. It is hard to cooperate in societies with greater heterogeneity. That is why the producing costs of the public good may depend on heterogeneity in society. The main conclusion will not change because higher costs will decrease the provided amount so that ethnic heterogeneity will have another channel of influence.

The result is robust only for parks because the effect is significant in OLS and propensity-score estimations. It is not robust for libraries and secondary education because of differences between general OLS regression controlling for fractionalization and propensity score dose-response estimation. A more complex robustness check requires more covariates, including unemployment, criminal level, population density, etc.  

The other possible concern is about different levels of aggregation. County's government may maximize the welfare of the county where the city is just a part of it. That is why citizens' preferences may not be taken into account by the county government. Hence, I may have a bias in my estimations. A possible solution for that is to control for a part of the costs funded by the county.

The other possible concern is the economic significance of the found results. It is hard to elaborate on the magnitude of the effect because it is hard to interpret my alienation index. The results are probably not economically significant, and the ethnic agenda is not important to policymakers. That is why future research is needed.


\section{Conclusion}

This is the first view on the impact of attitudes towards ethnic diversity on the public good provision. Further research may include better calculation of alienation and construction of attitudes of every group to every group. Surveys about attitudes towards heterogeneity may construct better alienation. The construction of attitudes of every group to every group may impact the results of the research because cities with general attitudes towards diversity seem not to suffer from ethnic tensions between two groups, while others are tolerant. That may influence all ways of life, including public good production.

\newpage

\bibliographystyle{apacite}
\bibliography{reference}

\newpage
\section*{Appendix}

\begin{table}[h]
    \centering
    \footnotesize
  {
\def\sym#1{\ifmmode^{#1}\else\(^{#1}\)\fi}
\begin{tabular}{l*{6}{c}}
\hline\hline
                    &\multicolumn{1}{c}{(1)}&\multicolumn{1}{c}{(2)}&\multicolumn{1}{c}{(3)}&\multicolumn{1}{c}{(4)}&\multicolumn{1}{c}{(5)}&\multicolumn{1}{c}{(6)}\\
                    &\multicolumn{1}{c}{Parks}&\multicolumn{1}{c}{Parks}&\multicolumn{1}{c}{Parks}&\multicolumn{1}{c}{Parks}&\multicolumn{1}{c}{Parks}&\multicolumn{1}{c}{Parks}\\
\hline
Ethnic fractionalization&     -0.0416\sym{**} &     -0.0354\sym{**} &                     &     -0.0445\sym{***}&     -0.0317\sym{*}  &                     \\
                    &    (0.0125)         &    (0.0133)         &                     &    (0.0128)         &    (0.0143)         &                     \\
[1em]
Alienation          &                     &     -0.0227         &      0.0630         &                     &     -0.0453\sym{*}  &      0.0352         \\
                    &                     &    (0.0175)         &    (0.0408)         &                     &    (0.0175)         &    (0.0414)         \\
[1em]
Index x Frac        &                     &                     &      -0.176\sym{**} &                     &                     &      -0.164\sym{*}  \\
                    &                     &                     &    (0.0615)         &                     &                     &    (0.0629)         \\
[1em]
Mean HH income      &                     &                     &                     &    5.82e-09         &    8.45e-09         &    9.02e-09         \\
                    &                     &                     &                     &  (7.06e-09)         &  (6.61e-09)         &  (6.91e-09)         \\
[1em]
Inequality          &                     &                     &                     &   -0.000830\sym{*}  &    -0.00106\sym{*}  &    -0.00111\sym{*}  \\
                    &                     &                     &                     &  (0.000403)         &  (0.000421)         &  (0.000427)         \\
[1em]
Log of city population&                     &                     &                     &     0.00443\sym{*}  &     0.00446\sym{*}  &     0.00449\sym{*}  \\
                    &                     &                     &                     &   (0.00215)         &   (0.00204)         &   (0.00199)         \\
[1em]
Taxes collected     &                     &                     &                     &-0.000000691         &-0.000000574         &-0.000000573         \\
                    &                     &                     &                     &(0.000000904)         &(0.000000805)         &(0.000000768)         \\
[1em]
Constant            &      0.0496\sym{***}&      0.0512\sym{***}&      0.0350\sym{***}&     0.00204         &     0.00520         &    -0.00964         \\
                    &   (0.00612)         &   (0.00637)         &   (0.00435)         &    (0.0289)         &    (0.0278)         &    (0.0283)         \\
\hline
Observations        &         900         &         900         &         900         &         900         &         900         &         900         \\
\hline\hline
\multicolumn{7}{l}{\footnotesize Standard errors in parentheses}\\
\multicolumn{7}{l}{\footnotesize \sym{*} \(p<0.05\), \sym{**} \(p<0.01\), \sym{***} \(p<0.001\)}\\
\end{tabular}
}

    \caption{Results on share spent on parks and recreational facilities}
    \label{tab: parks}
\end{table}

\begin{table}[h]
    \centering
    \footnotesize
  {
\def\sym#1{\ifmmode^{#1}\else\(^{#1}\)\fi}
\begin{tabular}{l*{6}{c}}
\hline\hline
                    &\multicolumn{1}{c}{(1)}&\multicolumn{1}{c}{(2)}&\multicolumn{1}{c}{(3)}&\multicolumn{1}{c}{(4)}&\multicolumn{1}{c}{(5)}&\multicolumn{1}{c}{(6)}\\
                    &\multicolumn{1}{c}{Libraries}&\multicolumn{1}{c}{Libraries}&\multicolumn{1}{c}{Libraries}&\multicolumn{1}{c}{Libraries}&\multicolumn{1}{c}{Libraries}&\multicolumn{1}{c}{Libraries}\\
\hline
Ethnic fractionalization&    -0.00957\sym{**} &    -0.00900\sym{**} &                     &    -0.00890\sym{**} &    -0.00808         &                     \\
                    &   (0.00286)         &   (0.00339)         &                     &   (0.00337)         &   (0.00412)         &                     \\
[1em]
Alienation          &                     &    -0.00207         &      0.0178         &                     &    -0.00290         &      0.0136         \\
                    &                     &   (0.00539)         &    (0.0121)         &                     &   (0.00575)         &    (0.0140)         \\
[1em]
Index x Frac        &                     &                     &     -0.0419\sym{*}  &                     &                     &     -0.0354         \\
                    &                     &                     &    (0.0167)         &                     &                     &    (0.0199)         \\
[1em]
Mean HH income      &                     &                     &                     &    2.60e-10         &    4.28e-10         &    7.70e-10         \\
                    &                     &                     &                     &  (2.01e-09)         &  (2.11e-09)         &  (2.05e-09)         \\
[1em]
Inequality          &                     &                     &                     &  -0.0000530         &  -0.0000680         &  -0.0000948         \\
                    &                     &                     &                     &  (0.000127)         &  (0.000138)         &  (0.000131)         \\
[1em]
Log of city population&                     &                     &                     &   -0.000210         &   -0.000208         &   -0.000249         \\
                    &                     &                     &                     &  (0.000505)         &  (0.000504)         &  (0.000502)         \\
[1em]
Taxes collected     &                     &                     &                     &   -2.08e-08         &   -1.33e-08         &   -4.11e-08         \\
                    &                     &                     &                     &(0.000000283)         &(0.000000284)         &(0.000000278)         \\
[1em]
Constant            &      0.0133\sym{***}&      0.0134\sym{***}&     0.00941\sym{***}&      0.0161\sym{*}  &      0.0163\sym{*}  &      0.0135         \\
                    &   (0.00141)         &   (0.00146)         &   (0.00129)         &   (0.00642)         &   (0.00649)         &   (0.00715)         \\
\hline
Observations        &         900         &         900         &         900         &         900         &         900         &         900         \\
\hline\hline
\multicolumn{7}{l}{\footnotesize Standard errors in parentheses}\\
\multicolumn{7}{l}{\footnotesize \sym{*} \(p<0.05\), \sym{**} \(p<0.01\), \sym{***} \(p<0.001\)}\\
\end{tabular}
}

    \caption{Results on share spent on libraries}
    \label{tab: libraries}
\end{table}

\begin{table}[h]
    \centering
    \footnotesize
  {
\def\sym#1{\ifmmode^{#1}\else\(^{#1}\)\fi}
\begin{tabular}{l*{6}{c}}
\hline\hline
                    &\multicolumn{1}{c}{(1)}&\multicolumn{1}{c}{(2)}&\multicolumn{1}{c}{(3)}&\multicolumn{1}{c}{(4)}&\multicolumn{1}{c}{(5)}&\multicolumn{1}{c}{(6)}\\
                    &\multicolumn{1}{c}{Education}&\multicolumn{1}{c}{Education}&\multicolumn{1}{c}{Education}&\multicolumn{1}{c}{Education}&\multicolumn{1}{c}{Education}&\multicolumn{1}{c}{Education}\\
\hline
Ethnic fractionalization&     -0.0750         &     -0.0349         &                     &      0.0623         &      0.0809         &                     \\
                    &    (0.0691)         &    (0.0704)         &                     &    (0.0681)         &    (0.0622)         &                     \\
[1em]
Alienation          &                     &      -0.147         &      0.0557         &                     &     -0.0656         &      -0.189         \\
                    &                     &     (0.149)         &     (0.270)         &                     &     (0.123)         &     (0.228)         \\
[1em]
Index x Frac        &                     &                     &      -0.360         &                     &                     &       0.288         \\
                    &                     &                     &     (0.400)         &                     &                     &     (0.411)         \\
[1em]
Mean HH income      &                     &                     &                     &    6.23e-08         &    6.61e-08         &    6.06e-08         \\
                    &                     &                     &                     &  (5.34e-08)         &  (5.38e-08)         &  (5.34e-08)         \\
[1em]
Inequality          &                     &                     &                     &    -0.00271         &    -0.00305         &    -0.00262         \\
                    &                     &                     &                     &   (0.00335)         &   (0.00333)         &   (0.00337)         \\
[1em]
Log of city population&                     &                     &                     &     -0.0271\sym{*}  &     -0.0270\sym{*}  &     -0.0261\sym{*}  \\
                    &                     &                     &                     &    (0.0108)         &    (0.0108)         &    (0.0108)         \\
[1em]
Taxes collected     &                     &                     &                     &  -0.0000316\sym{**} &  -0.0000314\sym{***}&  -0.0000308\sym{**} \\
                    &                     &                     &                     &(0.00000933)         &(0.00000920)         &(0.00000909)         \\
[1em]
Constant            &       0.389\sym{***}&       0.399\sym{***}&       0.377\sym{***}&       0.732\sym{***}&       0.737\sym{***}&       0.756\sym{***}\\
                    &    (0.0282)         &    (0.0310)         &    (0.0273)         &     (0.129)         &     (0.131)         &     (0.145)         \\
\hline
Observations        &         900         &         900         &         900         &         900         &         900         &         900         \\
\hline\hline
\multicolumn{7}{l}{\footnotesize Standard errors in parentheses}\\
\multicolumn{7}{l}{\footnotesize \sym{*} \(p<0.05\), \sym{**} \(p<0.01\), \sym{***} \(p<0.001\)}\\
\end{tabular}
}

    \caption{Results on share spent on secondary education}
    \label{tab: education}
\end{table}

\begin{table}[h]
    \centering
    \footnotesize
  {
\def\sym#1{\ifmmode^{#1}\else\(^{#1}\)\fi}
\begin{tabular}{l*{6}{c}}
\hline\hline
                    &\multicolumn{1}{c}{(1)}&\multicolumn{1}{c}{(2)}&\multicolumn{1}{c}{(3)}&\multicolumn{1}{c}{(4)}&\multicolumn{1}{c}{(5)}&\multicolumn{1}{c}{(6)}\\
                    &\multicolumn{1}{c}{Police}&\multicolumn{1}{c}{Police}&\multicolumn{1}{c}{Police}&\multicolumn{1}{c}{Police}&\multicolumn{1}{c}{Police}&\multicolumn{1}{c}{Police}\\
\hline
Ethnic fractionalization&     -0.0130         &     -0.0137         &                     &     0.00348         &     0.00950         &                     \\
                    &    (0.0224)         &    (0.0272)         &                     &    (0.0234)         &    (0.0265)         &                     \\
[1em]
Alienation          &                     &     0.00290         &      0.0927         &                     &     -0.0212         &      0.0244         \\
                    &                     &    (0.0373)         &     (0.105)         &                     &    (0.0342)         &    (0.0993)         \\
[1em]
Index x Frac        &                     &                     &      -0.157         &                     &                     &     -0.0627         \\
                    &                     &                     &     (0.150)         &                     &                     &     (0.145)         \\
[1em]
Mean HH income      &                     &                     &                     &    2.97e-08         &    3.09e-08         &    2.74e-08         \\
                    &                     &                     &                     &  (1.64e-08)         &  (1.62e-08)         &  (1.59e-08)         \\
[1em]
Inequality          &                     &                     &                     &    -0.00226\sym{**} &    -0.00237\sym{**} &    -0.00209\sym{**} \\
                    &                     &                     &                     &  (0.000854)         &  (0.000824)         &  (0.000792)         \\
[1em]
Log of city population&                     &                     &                     &     0.00473         &     0.00475         &     0.00556         \\
                    &                     &                     &                     &   (0.00348)         &   (0.00350)         &   (0.00334)         \\
[1em]
Taxes collected     &                     &                     &                     & -0.00000659\sym{***}& -0.00000653\sym{***}& -0.00000605\sym{***}\\
                    &                     &                     &                     &(0.00000172)         &(0.00000168)         &(0.00000160)         \\
[1em]
Constant            &       0.139\sym{***}&       0.139\sym{***}&       0.130\sym{***}&      0.0935\sym{*}  &      0.0950\sym{*}  &      0.0838\sym{*}  \\
                    &    (0.0107)         &    (0.0103)         &   (0.00846)         &    (0.0403)         &    (0.0398)         &    (0.0413)         \\
\hline
Observations        &         900         &         900         &         900         &         900         &         900         &         900         \\
\hline\hline
\multicolumn{7}{l}{\footnotesize Standard errors in parentheses}\\
\multicolumn{7}{l}{\footnotesize \sym{*} \(p<0.05\), \sym{**} \(p<0.01\), \sym{***} \(p<0.001\)}\\
\end{tabular}
}

    \caption{Results on share spent on police}
    \label{tab: police}
\end{table}

\begin{table}[h]
    \centering
    \footnotesize
  {
\def\sym#1{\ifmmode^{#1}\else\(^{#1}\)\fi}
\begin{tabular}{l*{6}{c}}
\hline\hline
                    &\multicolumn{1}{c}{(1)}&\multicolumn{1}{c}{(2)}&\multicolumn{1}{c}{(3)}&\multicolumn{1}{c}{(4)}&\multicolumn{1}{c}{(5)}&\multicolumn{1}{c}{(6)}\\
                    &\multicolumn{1}{c}{Welfare}&\multicolumn{1}{c}{Welfare}&\multicolumn{1}{c}{Welfare}&\multicolumn{1}{c}{Welfare}&\multicolumn{1}{c}{Welfare}&\multicolumn{1}{c}{Welfare}\\
\hline
Ethnic fractionalization&      0.0581\sym{*}  &     0.00117         &                     &     0.00134         &     -0.0546\sym{*}  &                     \\
                    &    (0.0278)         &    (0.0254)         &                     &    (0.0260)         &    (0.0228)         &                     \\
[1em]
Alienation          &                     &       0.209\sym{***}&       0.172\sym{*}  &                     &       0.197\sym{***}&       0.310\sym{***}\\
                    &                     &    (0.0329)         &    (0.0808)         &                     &    (0.0342)         &    (0.0744)         \\
[1em]
Index x Frac        &                     &                     &      0.0599         &                     &                     &      -0.242         \\
                    &                     &                     &     (0.135)         &                     &                     &     (0.126)         \\
[1em]
Mean HH income      &                     &                     &                     &   -2.31e-08         &   -3.46e-08\sym{*}  &   -3.24e-08\sym{*}  \\
                    &                     &                     &                     &  (1.41e-08)         &  (1.33e-08)         &  (1.31e-08)         \\
[1em]
Inequality          &                     &                     &                     &     0.00139         &     0.00240\sym{*}  &     0.00223\sym{*}  \\
                    &                     &                     &                     &   (0.00104)         &  (0.000981)         &  (0.000935)         \\
[1em]
Log of city population&                     &                     &                     &    -0.00308         &    -0.00324         &    -0.00350         \\
                    &                     &                     &                     &   (0.00426)         &   (0.00389)         &   (0.00387)         \\
[1em]
Taxes collected     &                     &                     &                     &   0.0000231\sym{***}&   0.0000226\sym{***}&   0.0000224\sym{***}\\
                    &                     &                     &                     &(0.00000455)         &(0.00000503)         &(0.00000501)         \\
[1em]
Constant            &     0.00894         &    -0.00567         &    -0.00340         &      0.0212         &     0.00753         &     -0.0123         \\
                    &    (0.0110)         &    (0.0115)         &   (0.00758)         &    (0.0523)         &    (0.0467)         &    (0.0463)         \\
\hline
Observations        &         900         &         900         &         900         &         900         &         900         &         900         \\
\hline\hline
\multicolumn{7}{l}{\footnotesize Standard errors in parentheses}\\
\multicolumn{7}{l}{\footnotesize \sym{*} \(p<0.05\), \sym{**} \(p<0.01\), \sym{***} \(p<0.001\)}\\
\end{tabular}
}

    \caption{Results on share spent on welfare}
    \label{tab: welfare}
\end{table}

\begin{table}[]
    \centering
    \footnotesize
  {
\def\sym#1{\ifmmode^{#1}\else\(^{#1}\)\fi}
\begin{tabular}{l*{6}{c}}
\hline\hline
                    &\multicolumn{1}{c}{(1)}&\multicolumn{1}{c}{(2)}&\multicolumn{1}{c}{(3)}&\multicolumn{1}{c}{(4)}&\multicolumn{1}{c}{(5)}&\multicolumn{1}{c}{(6)}\\
                    &\multicolumn{1}{c}{Social}&\multicolumn{1}{c}{Social}&\multicolumn{1}{c}{Social}&\multicolumn{1}{c}{Social}&\multicolumn{1}{c}{Social }&\multicolumn{1}{c}{Social}\\
\hline
Ethnic fractionalization&       0.163\sym{**} &      0.0524         &                     &      0.0632         &      0.0117         &                     \\
                    &    (0.0515)         &    (0.0865)         &                     &    (0.0499)         &    (0.0548)         &                     \\
[1em]
Index x Frac        &                     &       0.364         &       0.617\sym{*}  &                     &                     &      0.0592         \\
                    &                     &     (0.210)         &     (0.285)         &                     &                     &     (0.268)         \\
[1em]
Alienation          &                     &                     &      -0.122         &                     &       0.181         &       0.152         \\
                    &                     &                     &     (0.212)         &                     &    (0.0970)         &     (0.192)         \\
[1em]
Mean HH income      &                     &                     &                     &-0.000000118         &-0.000000128         &-0.000000129         \\
                    &                     &                     &                     &  (6.88e-08)         &  (6.97e-08)         &  (7.06e-08)         \\
[1em]
Inequality          &                     &                     &                     &     0.00573         &     0.00667         &     0.00669         \\
                    &                     &                     &                     &   (0.00431)         &   (0.00443)         &   (0.00449)         \\
[1em]
Log of city population&                     &                     &                     &     0.00717         &     0.00702         &     0.00702         \\
                    &                     &                     &                     &    (0.0110)         &    (0.0111)         &    (0.0111)         \\
[1em]
Taxes collected     &                     &                     &                     &   0.0000270\sym{***}&   0.0000265\sym{***}&   0.0000265\sym{***}\\
                    &                     &                     &                     &(0.00000504)         &(0.00000514)         &(0.00000503)         \\
[1em]
Constant            &      0.0372         &      0.0538\sym{*}  &      0.0776\sym{**} &     -0.0589         &     -0.0715         &     -0.0663         \\
                    &    (0.0188)         &    (0.0212)         &    (0.0242)         &     (0.131)         &     (0.132)         &     (0.138)         \\
\hline
Observations        &         900         &         900         &         900         &         900         &         900         &         900         \\
\hline\hline
\multicolumn{7}{l}{\footnotesize Standard errors in parentheses}\\
\multicolumn{7}{l}{\footnotesize \sym{*} \(p<0.05\), \sym{**} \(p<0.01\), \sym{***} \(p<0.001\)}\\
\end{tabular}
}

    \caption{Results on share spent on social service}
    \label{tab: social service}
\end{table}





\end{document}
