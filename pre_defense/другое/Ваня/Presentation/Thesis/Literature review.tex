Ethnic diversity is an essential determinant of economic performance. For example, \cite{Africa} shows that ethnic diversity explains the difference in public policies between countries in Sub-Saharan Africa. These varieties explain lower economic growth in countries with higher ethnic fragmentation. 

Talking about public goods, the classical paper from \cite{AlesinaDivision} argues that the higher ethnic diversity is associated with the lower provision of some public goods. They presented a model in which different ethnic groups have different preferences for the public good to explain such dependence. They have found that spending on education, roads, sewers, and trash pickup is inversely related to fractionalization index, while police, health, and welfare had no effect. The endogeneity concern occurs here because the data was from US municipalities, and people may move between them. The paper from \cite{Kenya} deals with it because the ethnic patterns are stable and historically determined. They showed that ethnic fractionalization is associated with lower spending on primary schools. In addition, they presented a model based on the other mechanism of social sanctions. Higher ethnic fractionalization causes a failure of collective actions due to incapacity to implement sanctions. In addition, \cite{Indonesia} support all results of previous papers. The authors use the variation in ethnic fragmentation because of new jurisdictions and show that greater ethnic fractionalization causes increased deforestation. This is consistent with the paper on under provision of public goods. Moreover, the \cite{transfer} have found the same effect but presented a different mechanism. They argue that higher ethnic diversity increases the cost of producing public goods. It can be explained by the greater transaction costs and strengthened the problem of public action.

\cite{WhyUndermine} contributed to the mechanism of collective action. They ran several experiments and have found that the strategy selection mechanism is the main one. In similar conditions, agents from the same ethnic group play cooperative equilibrium, but agents from different groups do not. At the same time, they argue that the preferences mechanism is not the same one. However, \cite{Cooperation} have different results. They argue that evolutionary game theory predicts deviation from cooperative strategy, but social diversity helps deal with this problem. To prove that, the authors introduced an experiment with a public good game and showed that diversity promotes cooperation.

\cite{difference} argues that previous papers focused on fractionalization while the distance in social and economic characteristics also matters. They have found that economic inequality strengthens the effect shown in previous papers.

At this moment, it is crucial to elaborate on different polarization and fractionalization indexes. Firstly, \cite{Reynal-Querol} introduced polarization index instead of fractionalization. They argue that previous papers have shown a weak connection between fractionalization and conflict that explains a lower economic growth rate. The authors develop a new polarization index and show that it is better to explain the frequency of civil wars. Secondly, \cite{EstebanRay} introduced their theory of the measurement of polarization. They presented the polarization index and used it for analysis of ethnic diversity. \cite{EstebanRayConflict} distinguished between the occurrence of the conflict and its seriousness. They have shown that fractionalization and polarization tend to have opposite effects on both aspects, and the second one has a positive impact on the intensity of the conflict. These facts suggest that attitude toward ethnic diversity has a higher impact than its fragmentation. Other paper from \cite{alpha} presents different the family of polarization indices that I will use in the paper.

Talking about an attitude towards diversity, \cite{Foreign} explain determinants of attitudes towards foreigners and shows significant variation in the period from 1988 to 2008. \cite{Refuges} show that greater ethnic fractionalization and polarization are associated with lower support of refugees in Europe, which means a worse attitude to diversity. At the same time, this effect is not robust to different measures of ethnic diversity. 

Literature leaves me with the idea that attitude towards ethnic diversity affects public goods provision instead of actual fractionalization.